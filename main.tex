\documentclass{article}
\usepackage[utf8]{inputenc}
\usepackage{fancyvrb}

\renewcommand{\labelenumii}{\theenumii.}
\newcommand{\be}{\begin{enumerate}}
\newcommand{\ee}{\end{enumerate}}
\newcommand{\bi}{\begin{itemize}}
\newcommand{\ei}{\end{itemize}}
\newcommand{\bc}{\begin{center}}
\newcommand{\ec}{\end{center}}
\newcommand{\bsp}{\begin{sloppypar}}
\newcommand{\esp}{\end{sloppypar}}

\newcommand{\sA}{{\cal A}}
\newcommand{\sB}{{\cal B}}
\newcommand{\sC}{{\cal C}}
\newcommand{\sD}{{\cal D}}
\newcommand{\sE}{{\cal E}}
\newcommand{\sF}{{\cal F}}
\newcommand{\sG}{{\cal G}}
\newcommand{\sH}{{\cal H}}
\newcommand{\sI}{{\cal I}}
\newcommand{\sJ}{{\cal J}}
\newcommand{\sK}{{\cal K}}
\newcommand{\sL}{{\cal L}}
\newcommand{\sM}{{\cal M}}
\newcommand{\sN}{{\cal N}}
\newcommand{\sO}{{\cal O}}
\newcommand{\sP}{{\cal P}}
\newcommand{\sQ}{{\cal Q}}
\newcommand{\sR}{{\cal R}}
\newcommand{\sS}{{\cal S}}
\newcommand{\sT}{{\cal T}}
\newcommand{\sU}{{\cal U}}
\newcommand{\sV}{{\cal V}}
\newcommand{\sW}{{\cal W}}
\newcommand{\sX}{{\cal X}}
\newcommand{\sY}{{\cal Y}}
\newcommand{\sZ}{{\cal Z}}

\newcommand{\bA}{\mathbb{A}}
\newcommand{\bB}{\mathbb{B}}
\newcommand{\bC}{\mathbb{C}}
\newcommand{\bD}{\mathbb{D}}
\newcommand{\bE}{\mathbb{E}}
\newcommand{\bF}{\mathbb{F}}
\newcommand{\bG}{\mathbb{G}}
\newcommand{\bH}{\mathbb{H}}
\newcommand{\bI}{\mathbb{I}}
\newcommand{\bJ}{\mathbb{J}}
\newcommand{\bK}{\mathbb{K}}
\newcommand{\bL}{\mathbb{L}}
\newcommand{\bM}{\mathbb{M}}
\newcommand{\bN}{\mathbb{N}}
\newcommand{\bO}{\mathbb{O}}
\newcommand{\bP}{\mathbb{P}}
\newcommand{\bQ}{\mathbb{Q}}
\newcommand{\bR}{\mathbb{R}}
\newcommand{\bS}{\mathbb{S}}
\newcommand{\bT}{\mathbb{T}}
\newcommand{\bU}{\mathbb{U}}
\newcommand{\bV}{\mathbb{V}}
\newcommand{\bW}{\mathbb{W}}
\newcommand{\bX}{\mathbb{X}}
\newcommand{\bY}{\mathbb{Y}}
\newcommand{\bZ}{\mathbb{Z}}


\newcommand{\sglsp}{\ }
\newcommand{\dblsp}{\ \ }
\newcommand{\textrel}[1]{\dblsp \text{#1} \dblsp}

\newcommand{\TRUE}{\mbox{{\sc t}}}
\newcommand{\FALSE}{\mbox{{\sc f}}}
\newcommand{\Neg}{\neg} 
\ifdefined \And 
\renewcommand{\And}{\wedge}
\else
\newcommand{\And}{\wedge}
\fi
\newcommand{\Or}{\vee}
\newcommand{\Implies}{\Rightarrow}
\newcommand{\Iff}{\Leftrightarrow}
\newcommand{\Forall}{\forall}
\newcommand{\ForallApp}{\forall\,}
\newcommand{\Forsome}{\exists}
\newcommand{\ForsomeApp}{\exists\,}
\newcommand{\ForsomeUnique}{\exists!}
\newcommand{\ForsomeUniqueApp}{\exists!\,}
\newcommand{\LambdaApp}{\lambda\,}
\newcommand{\LAMBDAapp}{\Lambda\,}
\newcommand{\PiApp}{\Pi\,}
\newcommand{\SigmaApp}{\Sigma\,}
\newcommand{\iotaApp}{\iota\,}
\newcommand{\Iota}{{\rm I}}
\newcommand{\IotaApp}{{\rm I}\,}
\newcommand{\invertediota}{\rotatebox[origin=c]{180}{$\iota$}}
\newcommand{\epsilonApp}{\epsilon\,}
\newcommand{\Set}{\circledS}
\newcommand{\SetApp}{\circledS\,}

\newcommand{\IsDef}{\downarrow}
\newcommand{\IsUndef}{\uparrow}
\newcommand{\IsDefApp}{\,{\IsDef}}
\newcommand{\IsUndefApp}{\,{\IsUndef}}
\newcommand{\QuasiEqual}{\simeq}
\newcommand{\Undefined}{\bot}
\newcommand{\If}[3]{#1 \mapsto #2 \mid #3}
\newcommand{\SynEqual}{\equiv}

\renewcommand{\phi}{\varphi}
%\newcommand{\set}[1]{{\{#1\}}}
\newcommand{\seq}[1]{{\langle #1 \rangle}}
\newcommand{\seqlike}[1]{\seq{\!\seq{#1}\!}}
%\newcommand{\mlist}[1]{[#1]}
\newcommand{\compl}[1]{\overline{#1}}
\newcommand{\recip}[1]{{#1}^{-1}}
\newcommand{\pow}[1]{\sP(#1)}
\newcommand{\tarrow}{\rightarrow}
%\newcommand{\bo}{\omicron}
\newcommand{\consext}{\trianglelefteq}
\newcommand{\mIff}{\textrel{iff}}
\newcommand{\mImp}{\textrel{implies}}
\newcommand{\mEq}{\dblsp{=}\dblsp}
\newcommand{\mNotEq}{\dblsp{\not=}\dblsp}
\newcommand{\mQEq}{\dblsp{\simeq}\dblsp}
\newcommand{\mNotQEq}{\dblsp{\not\simeq}\dblsp}
\newcommand{\mIncluded}{\preceq}
\newcommand{\hr}{\phantom{}^\ast\mathbb{R}}
\newcommand{\restricted}{\mbox{$\upharpoonright$}\nolinebreak}
\newcommand{\subfun}{\sqsubseteq}
\newcommand{\clexpr}[1]{\overline{\sE_{#1}}}

\newcommand{\churchqe}{$\mbox{\sc ctt}_{\rm qe}$}
\newcommand{\churchuqe}{$\mbox{\sc ctt}_{\rm uqe}$}
\newcommand{\qzero}{${\cal Q}_0$}
\newcommand{\qzerou}{${\cal Q}^{\rm u}_{0}$}
\newcommand{\qzerouqe}{${\cal Q}^{\rm uqe}_{0}$}
\newcommand{\fstar}{${\rm F}^\ast$}

\newenvironment{proof}[1][\unskip]{\par\noindent{\bf Proof{#1}\dblsp}}{\hfill$\Box$}

\newcommand{\urlpart}[1]{\mbox{\texttt{#1}}\linebreak[0]}

%% LaTeX Macros for Alonzo Notation
%%
%% William M. Farmer
%%
%% McMaster University

%% Requires:

\usepackage{amssymb}
\usepackage{amsmath}
\usepackage{phonetic}
\usepackage[dvipsnames]{xcolor}

%%%%%%%%%%%%%%%%%%%%%%%%%%%%%%%%%%%%%%%%%%%%%%%%%%%%%%%%%%%%%%%%%

% MISCELLANEOUS MACROS

\newcommand{\mName}[1]{\mathsf{#1}}
\newcommand{\mSet}[1]{\{ #1 \}}
\newcommand{\mTuple}[1]{( #1 )}
\newcommand{\mList}[1]{[ #1 ]}
\newcommand{\mSeq}[1]{\langle #1 \rangle}
\newcommand{\mSeqlike}[1]{\mSeq{\!\mSeq{#1}\!}}
\newcommand{\mAbs}[1]{\lvert #1 \rvert}
\newcommand{\mNorm}[1]{\lVert #1 \rVert}
\newcommand{\mDot}{\mathrel{.}}

%%%%%%%%%%%%%%%%%%%%%%%%%%%%%%%%%%%%%%%%%%%%%%%%%%%%%%%%%%%%%%%%%

% FORMAL NOTATION

% Types

\newcommand{\fBoolTy}{\mName{BoolTy}}
\newcommand{\fBaseTy}[1]{\mName{BaseTy}(#1)}
\newcommand{\fFunTy}[2]{\mName{FunTy}(#1,#2)}
\newcommand{\fProdTy}[2]{\mName{ProdTy}(#1,#2)}

% Expressions

\newcommand{\fVar}[2]{\mName{Var}(#1,#2)}
\newcommand{\fCon}[2]{\mName{Con}(#1,#2)}
\newcommand{\fEq}[2]{\mName{Eq}(#1,#2)}
\newcommand{\fFunApp}[2]{\mName{FunApp}(#1,#2)}
\newcommand{\fFunAbs}[3]{\mName{FunAbs}(\fVar{#1}{#2},#3)}
\newcommand{\fDefDes}[3]{\mName{DefDes}(\fVar{#1}{#2},#3)}
\newcommand{\fOrdPair}[2]{\mName{OrdPair}(#1,#2)}

%%%%%%%%%%%%%%%%%%%%%%%%%%%%%%%%%%%%%%%%%%%%%%%%%%%%%%%%%%%%%%%%%

% COMPACT NOTATION

% Types

\newcommand{\cBoolTy}{\omicron}
\newcommand{\cB}{\cBoolTy}
\newcommand{\cBaseTy}[1]{#1}
\newcommand{\cFunTy}[2]{({#1} \rightarrow {#2})}
\newcommand{\cFunTyX}[2]{{#1} \rightarrow {#2}}
\newcommand{\cFunTyB}[3]{\cFunTy {#1} {\cFunTyX {#2} {#3}}}
\newcommand{\cFunTyBX}[3]{\cFunTyX {#1} {\cFunTyX {#2} {#3}}}
\newcommand{\cFunTyC}[4]{\cFunTy {#1} {\cFunTyX {#2} {\cFunTyX {#3} {#4}}}}
\newcommand{\cFunTyCX}[4]{\cFunTyX {#1} {\cFunTyX {#2} {\cFunTyX {#3} {#4}}}}
\newcommand{\cProdTy}[2]{({#1} \times {#2})}
\newcommand{\cProdTyX}[2]{{#1} \times {#2}}
\newcommand{\cProdTyB}[3]{\cProdTy {#1} {\cProdTyX {#2} {#3}}}
\newcommand{\cProdTyBX}[3]{\cProdTyX {#1} {\cProdTyX {#2} {#3}}}
\newcommand{\cProdTyC}[4]{\cProdTy {#1} {\cProdTyX {#2} {\cProdTyX {#3} {#4}}}}
\newcommand{\cProdTyCX}[4]{\cProdTyX {#1} {\cProdTyX {#2} {\cProdTyX {#3} {#4}}}}

% Expressions

\newcommand{\cVar}[2]{({#1} : {#2})}
\newcommand{\cVarY}[2]{#1}
\newcommand{\cCon}[2]{{#1}_{#2}}
\newcommand{\cConY}[2]{#1}
\newcommand{\cEq}[2]{({#1} = {#2})}
\newcommand{\cEqX}[2]{{#1} = {#2}}
\newcommand{\cFunApp}[2]{(#1\,#2)}
\newcommand{\cFunAppX}[2]{#1\,#2}
\newcommand{\cFunAppB}[3]{(\cFunAppX {\cFunAppX {#1} {#2}} {#3})}
\newcommand{\cFunAppBX}[3]{\cFunAppX {\cFunAppX {#1} {#2}} {#3}}
\newcommand{\cFunAppC}[4]{(\cFunAppX {\cFunAppX {\cFunAppX {#1}{#2}}{#3}}{#4})}
\newcommand{\cFunAppCX}[4]{\cFunAppX {\cFunAppX {\cFunAppX {#1}{#2}}{#3}}{#4}}
\newcommand{\cFunAbs}[3]{(\lambda\, #1 : #2 \mDot #3)}
\newcommand{\cFunAbsX}[3]{\lambda\, #1 : #2 \mDot #3}
\newcommand{\cDefDes}[3]{(\mathrm{I}\, #1 : #2 \mDot #3)}
\newcommand{\cDefDesX}[3]{\mathrm{I}\, #1 : #2 \mDot #3}
\newcommand{\cOrdPair}[2]{(#1,#2)}

% Boolean Operators

\newcommand{\cTPC}{T_\cB}
\newcommand{\cT}{\cTPC}
\newcommand{\cFPC}{F_\cB}
\newcommand{\cF}{\cFPC}
\newcommand{\cAndPC}{\wedge_{\cFunTyBX{\cB}{\cB}{\cB}}}
\newcommand{\cAnd}[2]{({#1} \wedge {#2})}
\newcommand{\cAndX}[2]{{#1} \wedge {#2}}
\newcommand{\cAndB}[3]{\cAnd {#1} {\cAndX {#2} {#3}}}
\newcommand{\cAndBX}[3]{\cAndX {#1} {\cAndX {#2} {#3}}}
\newcommand{\cAndL}[1]{(#1)}   % separator is $\And$
\newcommand{\cAndLX}[1]{#1}    % separator is $\And$
\newcommand{\cImpliesPC}{\Rightarrow_{\cFunTyBX {\cB} {\cB} {\cB}}}
\newcommand{\cImplies}[2]{({#1} \Rightarrow {#2})}
\newcommand{\cImpliesX}[2]{{#1} \Rightarrow {#2}}
\newcommand{\cNegPC}{\neg_{\cFunTyX{\cB}{\cB}}}
\newcommand{\cNeg}[1]{(\neg{#1})}
\newcommand{\cNegX}[1]{\neg{#1}}
\newcommand{\cOrPC}{\vee_{\cFunTyBX{\cB}{\cB}{\cB}}}
\newcommand{\cOr}[2]{({#1} \vee {#2})}
\newcommand{\cOrX}[2]{{#1} \vee {#2}}
\newcommand{\cOrB}[3]{\cOr {#1} {\cOrX {#2} {#3}}}
\newcommand{\cOrBX}[3]{\cOrX {#1} {\cOrX {#2} {#3}}}
\newcommand{\cOrL}[1]{(#1)}    % separator is $\Or$
\newcommand{\cOrLX}[1]{#1}     % separator is $\Or$
\newcommand{\cIfPC}[1]{\mName{if}_{\cFunTyCX{\cB}{#1}{#1}{#1}}}
\newcommand{\cIf}[3]{(#1 \mapsto #2 \mid #3)}
\newcommand{\cIfX}[3]{#1 \mapsto #2 \mid #3}

% Binary Operators

\newcommand{\cBin}[3]{({#1} \mathrel{#2} {#3})}
\newcommand{\cBinX}[3]{{#1} \mathrel{#2} {#3}}
\newcommand{\cBinB}[5]{({#1} \mathrel{#2} {#3} \mathrel{#4} {#5})}
\newcommand{\cBinBX}[5]{{#1} \mathrel{#2} {#3} \mathrel{#4} {#5}}
\newcommand{\cIff}[2]{({#1} \Leftrightarrow {#2})}
\newcommand{\cIffX}[2]{{#1} \Leftrightarrow {#2}}
\newcommand{\cNotEq}[2]{({#1} \not= {#2})}
\newcommand{\cNotEqX}[2]{{#1} \not= {#2}}

% Quantifiers

\newcommand{\cForall}[3]{(\forall\, #1 : #2 \mDot #3)}
\newcommand{\cForallX}[3]{\forall\, #1 : #2 \mDot #3}
\newcommand{\cForallB}[5]{(\forall\, #1 : #2,\, #3 : #4 \mDot #5)}
\newcommand{\cForallBX}[5]{\forall\, #1 : #2,\, #3 : #4 \mDot #5}
\newcommand{\cForallC}[7]{(\forall\, #1 : #2,\, #3 : #4,\, #5 : #6 \mDot #7)}
\newcommand{\cForallCX}[7]{\forall\, #1 : #2,\, #3 : #4,\, #5 : #6 \mDot #7}
\newcommand{\cForsome}[3]{(\exists\, #1 : #2 \mDot #3)}
\newcommand{\cForsomeX}[3]{\exists\, #1 : #2 \mDot #3}
\newcommand{\cForsomeB}[5]{(\exists\, #1 : #2,\, #3 : #4 \mDot #5)}
\newcommand{\cForsomeBX}[5]{\exists\, #1 : #2,\, #3 : #4 \mDot #5}
\newcommand{\cForsomeC}[7]{(\exists\, #1 : #2,\, #3 : #4,\, #5 : #6 \mDot #7)}
\newcommand{\cForsomeCX}[7]{\exists\, #1 : #2,\, #3 : #4,\, #5 : #6 \mDot #7}
\newcommand{\cForsomeUnique}[3]{(\exists!\, #1 : #2 \mDot #3)}
\newcommand{\cForsomeUniqueX}[3]{\exists!\, #1 : #2 \mDot #3}
\newcommand{\cForsomeUniqueB}[5]{(\exists!\, #1 : #2, \, #3 : #4 \mDot #5)}
\newcommand{\cForsomeUniqueBX}[5]{\exists!\, #1 : #2, \, #3 : #4 \mDot #5}
\newcommand{\cForsomeUniqueC}[7]{(\exists!\, #1 : #2, \, #3 : #4, \, #5 : #6     \mDot #7)}
\newcommand{\cForsomeUniqueCX}[7]{\exists!\, #1 : #2, \, #3 : #4, \, #5 : #6     \mDot #7}
% Definedness

\newcommand{\cBotPC}[1]{\bot_{#1}}
\newcommand{\cEmpFunPC}[2]{\Delta_{\cFunTyX {#1} {#2}}}
\newcommand{\cIsDef}[1]{(#1{\downarrow})}
\newcommand{\cIsDefX}[1]{#1{\downarrow}}
\newcommand{\cIsUndef}[1]{(#1{\uparrow})}
\newcommand{\cIsUndefX}[1]{#1{\uparrow}}
\newcommand{\cQuasiEq}[2]{({#1} \simeq {#2})}
\newcommand{\cQuasiEqX}[2]{{#1} \simeq {#2}}
\newcommand{\cNotQuasiEq}[2]{({#1} \not\simeq {#2})}
\newcommand{\cNotQuasiEqX}[2]{{#1} \not\simeq {#2}}

% Sets

\newcommand{\cSetTy}[1]{\mSet{#1}}
\newcommand{\cIn}[2]{({#1} \in {#2})}
\newcommand{\cInX}[2]{{#1} \in {#2}}
\newcommand{\cNotIn}[2]{({#1} \not\in {#2})}
\newcommand{\cNotInX}[2]{{#1} \not\in {#2}}
\newcommand{\cSet}[3]{\mSet{{{#1} : {#2}} \mid {#3}}}
\newcommand{\cEmpSetPC}[1]{\emptyset_{\cSetTy {#1}}}
\newcommand{\cEmpSetAltPC}[1]{\mSet{\,}_{\cSetTy {#1}}}
\newcommand{\cUnivSetPC}[1]{U_{\cSetTy {#1}}}
\newcommand{\cFinSet}[2]{\textsf{{$#1$}-{$#2$}-SET}}
\newcommand{\cFinSetL}[1]{\mSet{#1}}  % separator is ","
\newcommand{\cSubseteqPC}[1]{\subseteq_{\cFunTyBX {\cSetTy #1} {\cSetTy #1} {\cB}}}
\newcommand{\cUnionPC}[1]{\cup_{\cFunTyBX {\cSetTy #1} {\cSetTy #1} {\cSetTy #1}}}
\newcommand{\cUnion}[2]{\cBin {#1} {\cup} {#2}}
\newcommand{\cUnionX}[2]{\cBinX {#1} {\cup} {#2}}
\newcommand{\cIntersPC}[1]{\cap_{\cFunTyBX {\cSetTy #1} {\cSetTy #1} {\cSetTy #1}}}
\newcommand{\cInters}[2]{\cBin {#1} {\cap} {#2}}
\newcommand{\cIntersX}[2]{\cBinX {#1} {\cap} {#2}}
\newcommand{\cComplPC}[1]{\overline{\,\cdot\,}_{\cFunTyX {\cSetTy #1} {\cSetTy #1}}}
\newcommand{\cCompl}[1]{\overline{#1}}

% Tuples

\newcommand{\cTupleTyL}[1]{(#1)}  % separator is $\times$
\newcommand{\cTupleL}[1]{(#1)}    % separator is ","
\newcommand{\cFstPC}[2]{\mName{fst}_{\cFunTyX {\cProdTy {#1} {#2}} {#1}}}
\newcommand{\cSndPC}[2]{\mName{snd}_{\cFunTyX {\cProdTy {#1} {#2}} {#2}}}

% Functions

\newcommand{\cDomPC}[2]{\mName{dom}_{\cFunTyX {\cFunTy {#1} {#2}} {\cSetTy {#1}}}}
\newcommand{\cRanPC}[2]{\mName{ran}_{\cFunTyX {\cFunTy {#1} {#2}} {\cSetTy {#2}}}}
\newcommand{\cSubfuneqPC}[2]{\subfun_{\cFunTyBX {\cFunTy {#1} {#2}} {\cFunTy {#1} {#2}} {\cB}}}
\newcommand{\cFunCompPC}[3]{\circ_{\cFunTyBX {\cFunTy {#1} {#2}} {\cFunTy {#2} {#3}} {\cFunTy {#1} {#3}}}}
\newcommand{\cFunComp}[2]{({#1} \circ {#2})}
\newcommand{\cFunCompX}[2]{#1 \circ {#2}}
\newcommand{\cRestrictPC}[2]{\restricted_{\cFunTyBX {\cFunTy {#1} {#2}} {\cSetTy {#1}} {\cFunTy {#1} {#2}}}}
\newcommand{\cRestrict}[2]{(#1 \restricted_{#2})}
\newcommand{\cRestrictX}[2]{#1 \restricted_{#2}}

% Miscellaneous Notation

\newcommand{\cTotal}[1]{\mName{TOTAL}(#1)}
\newcommand{\cTotalB}[1]{\mName{TOTAL2}(#1)}
\newcommand{\cSurj}[1]{\mName{SURJ}(#1)}
\newcommand{\cSurjB}[1]{\mName{SURJ2}(#1)}
\newcommand{\cInj}[1]{\mName{INJ}(#1)}
\newcommand{\cBij}[1]{\mName{BIJ}(#1)}
\newcommand{\cDistinctL}[1]{\mName{DISTINCT}(#1)}  % separator is ","

% Quasitypes

\newcommand{\cFunAbsQTy}[3]{\cFunAbs {#1} {#2} {#3}}
\newcommand{\cFunAbsQTyX}[3]{\cFunAbsX {#1} {#2} {#3}}
\newcommand{\cForallQTy}[3]{\cForall {#1} {#2} {#3}}
\newcommand{\cForallQTyX}[3]{\cForallX {#1} {#2} {#3}}
\newcommand{\cForallQTyB}[5]{\cForallB {#1} {#2} {#3} {#4} {#5}}
\newcommand{\cForallQTyBX}[5]{\cForallBX {#1} {#2} {#3} {#4} {#5}}
\newcommand{\cForsomeQTy}[3]{\cForsome {#1} {#2} {#3}}
\newcommand{\cForsomeQTyX}[3]{\cForsomeX {#1} {#2} {#3}}
\newcommand{\cForsomeQTyB}[5]{\cForsomeB {#1} {#2} {#3} {#4} {#5}}
\newcommand{\cForsomeQTyBX}[5]{\cForsomeBX {#1} {#2} {#3} {#4} {#5}}
\newcommand{\cDefDesQTy}[3]{\cDefDes {#1} {#2} {#3}}
\newcommand{\cDefDesQTyX}[3]{\cDefDesX {#1} {#2} {#3}}
\newcommand{\cIsDefInQTy}[2]{({#1} \downarrow {#2})}
\newcommand{\cIsDefInQTyX}[2]{{#1} \downarrow {#2}}
\newcommand{\cIsUndefInQTy}[2]{({#1} \uparrow {#2})}
\newcommand{\cIsUndefInQTyX}[2]{{#1} \uparrow {#2}}
\newcommand{\cFunQTyPC}[2]{\rightarrow_{\cFunTyBX {\cSetTy {#1}} {\cSetTy {#2}} {\cSetTy {\cFunTyX {#1} {#2}}}}}
\newcommand{\cFunQTy}[2]{\cFunTy {#1} {#2}}
\newcommand{\cFunQTyX}[2]{\cFunTyX {#1} {#2}}
\newcommand{\cFunQTyB}[3]{\cFunTyB {#1} {#2} {#3}}
\newcommand{\cFunQTyBX}[3]{\cFunTyBX {#1} {#2} {#3}}
\newcommand{\cFunQTyC}[4]{\cFunTyC {#1} {#2} {#3} {#4}}
\newcommand{\cFunQTyCX}[4]{\cFunTyCX {#1} {#2} {#3} {#4}}
\newcommand{\cProdQTyPC}[2]{\times_{\cFunTyBX {\cSetTy {#1}} {\cSetTy {#2}} {\cSetTy {\cProdTyX {#1} {#2}}}}}
\newcommand{\cProdQTy}[2]{\cProdTy {#1} {#2}}
\newcommand{\cProdQTyX}[2]{\cProdTyX {#1} {#2}}
\newcommand{\cProdQTyB}[3]{\cProdTyB {#1} {#2} {#3}}
\newcommand{\cProdQTyBX}[3]{\cProdTyBX {#1} {#2} {#3}}
\newcommand{\cProdQTyC}[4]{\cProdTyC {#1} {#2} {#3} {#4}}
\newcommand{\cProdQTyCX}[4]{\cProdTyCX {#1} {#2} {#3} {#4}}
\newcommand{\cTotalOn}[3]{\textsf{TOTAL-ON}(#1,#2,#3)}
\newcommand{\cTotalOnB}[4]{\textsf{TOTAL-ON2}(#1,#2,#3,#4)}
\newcommand{\cSurjOn}[3]{\textsf{SURJ-ON}(#1,#2,#3)}
\newcommand{\cSurjOnB}[4]{\textsf{SURJ-ON2}(#1,#2,#3,#4)}
\newcommand{\cInjOn}[2]{\textsf{INJ-ON}(#1,#2)}
\newcommand{\cBijOn}[3]{\textsf{BIJ-ON}(#1,#2,#3)}
\newcommand{\cInf}[1]{\mName{INF}(#1)}
\newcommand{\cFin}[1]{\mName{FIN}(#1)}
\newcommand{\cCount}[1]{\mName{COUNT}(#1)}

% Dependent Quasitypes

\newcommand{\cPiTy}[2]{\cFunTyBX {\cSetTy {#1}} {\cFunTy {#1} {\cSetTy {#2}}} {\cSetTy {\cFunTyX {#1} {#2}}}}
\newcommand{\cPiPC}[2]{\Pi_{\cPiTy {#1} {#2}}}
\newcommand{\cPi}[3]{(\Pi\, #1 : #2 \mDot #3)}
\newcommand{\cPiX}[3]{\Pi\, #1 : #2 \mDot #3}
\newcommand{\cSigmaTy}[2]{\cFunTyBX {\cSetTy {#1}} {\cFunTy {#1} {\cSetTy {#2}}} {\cSetTy {\cProdTyX {#1} {#2}}}}
\newcommand{\cSigmaPC}[2]{\Sigma_{\cSigmaTy {#1} {#2}}}
\newcommand{\cSigma}[3]{(\Sigma\, #1 : #2 \mDot #3)}
\newcommand{\cSigmaX}[3]{\Sigma\, #1 : #2 \mDot #3}

% Sequences

\newcommand{\cSequencesPC}[2]{\mName{sequences}_{\cSetTy {\cFunTyX {#1} {#2}}}}
\newcommand{\cSeqQTy}[1]{\mSeqlike{#1}}
\newcommand{\cStreamsPC}[2]{\mName{streams}_{\cSetTy {\cFunTyX {#1} {#2}}}}
\newcommand{\cSeqInfQTy}[1]{\mSeq{#1}}
\newcommand{\cListsPC}[2]{\mName{lists}_{\cSetTy {\cFunTyX {#1} {#2}}}}
\newcommand{\cSeqFinQTy}[1]{\mList{#1}}
\newcommand{\cConsPC}[2]{\mName{cons}_{\cFunTyBX {#2} {\cFunTy {#1} {#2}} {\cFunTy {#1} {#2}}}}
\newcommand{\cCons}[2]{({#1} :: {#2})}
\newcommand{\cConsX}[2]{{#1} :: {#2}}
\newcommand{\cHdPC}[2]{\mName{hd}_{\cFunTyX {\cFunTy {#1} {#2}} {#2}}}
\newcommand{\cTlPC}[2]{\mName{tl}_{\cFunTyX {\cFunTy {#1} {#2}} {\cFunTy {#1} {#2}}}}
\newcommand{\cNilPC}[2]{\mName{nil}_{\cFunTyX {#1} {#2}}}
\newcommand{\cEmpListPC}[2]{{\mList{\;}}_{\cFunTyX {#1} {#2}}}
\newcommand{\cListL}[1]{\mList{#1}}  % separator is ","
\newcommand{\cLenPC}[2]{\mName{len}_{\cFunTyX {\cFunTy {#1} {#2}} {#1}}}
\newcommand{\cLen}{\mAbs}
\newcommand{\cNlistsPC}[2]{\mName{nlists}_{\cFunTyX {#1} {\cSetTy {\cFunTyX {#1} {#2}}}}}
\newcommand{\cSeqNFinQTy}[2]{\mList{#1}^{#2}}

% Real Numbers

\newcommand{\cRecip}{\recip}
\newcommand{\cFrac}{\frac}
\newcommand{\cAbs}{\mAbs}
\newcommand{\cSqrt}{\sqrt}
\newcommand{\cNorm}{\mNorm}
\newcommand{\cSum}[4]{\sum\limits_{{#1} = {#2}}^{#3} {#4}}
\newcommand{\cProd}[4]{\prod\limits_{{#1} = {#2}}^{#3} {#4}}
\newcommand{\cLim}[3]{\lim\limits_{{#1} \to {#2}} {#3}}
\newcommand{\cRightLim}[3]{\lim\limits_{{#1} \to {#2}^+} {#3}}
\newcommand{\cLeftLim}[3]{\lim\limits_{{#1} \to {#2}^-} {#3}}
\newcommand{\cLimSeq}[2]{\lim\limits_{{#1} \to \infty} {#2}}
\newcommand{\cIntegral}[4]{\int_{#1}^{#2} {#3}\,d{#4}}

%%%%%%%%%%%%%%%%%%%%%%%%%%%%%%%%%%%%%%%%%%%%%%%%%%%%%%%%%%%%%%%%%

% THEOREM ENVIRONMENTS

\newtheorem{thm}{Theorem}[section]
\newtheorem{cor}[thm]{Corollary}
\newtheorem{lem}[thm]{Lemma}
\newtheorem{prop}[thm]{Proposition}
\newtheorem{eg}[thm]{Example}
\newtheorem{rem}[thm]{Remark}

\newtheorem{thydef}[thm]{Theory Definition}
\newtheorem{thyext}[thm]{Theory Extension}
\newtheorem{devdef}[thm]{Development Definition}
\newtheorem{devext}[thm]{Development Extension}
\newtheorem{thytransdef}[thm]{Theory Translation Definition}
\newtheorem{thytransext}[thm]{Theory Translation Extension}
\newtheorem{devtransdef}[thm]{Development Translation Definition}
\newtheorem{devtransext}[thm]{Development Translation Extension}
\newtheorem{deftransport}[thm]{Definition Transportation}
\newtheorem{thmtransport}[thm]{Theorem Transportation}

%%%%%%%%%%%%%%%%%%%%%%%%%%%%%%%%%%%%%%%%%%%%%%%%%%%%%%%%%%%%%%%%%

% ENVIRONMENTS

\newenvironment{theory-def}[6]
{
\color{brown}
\begin{thydef}[#1]\em
\noindent
\begin{itemize} \setlength{\itemsep}{0pt}
\item[]\hspace{-3ex}\textbf{Name:} #2
\item[]\hspace{-3ex}\textbf{Base types:} #3
\item[]\hspace{-3ex}\textbf{Constants:} #4
\item[]\hspace{-3ex}\textbf{Axioms:} #5
\end{itemize}
 #6
\end{thydef} 
}

\newenvironment{theory-ext}[7]
{
\color{brown}
\begin{thyext}[#1]\em
\noindent
\begin{itemize} \setlength{\itemsep}{0pt}
\item[]\hspace{-3ex}\textbf{Name:} #2
\item[]\hspace{-3ex}\textbf{Extends\ } #3
\item[]\hspace{-3ex}\textbf{New base types:} #4
\item[]\hspace{-3ex}\textbf{New constants:} #5
\item[]\hspace{-3ex}\textbf{New axioms:} #6
\end{itemize}
#7
\end{thyext}
}

\newenvironment{dev-def}[4]
{
\color{brown}
\begin{devdef}[#1]\em
\noindent
\begin{itemize}
\item[]\hspace{-3ex}\textbf{Name:} #2
\item[]\hspace{-3ex}\textbf{Bottom theory:} #3
\item[]\hspace{-3ex}\textbf{Definitions and theorems:}
\end{itemize}
#4
\end{devdef}
}

\newenvironment{dev-ext}[4]
{
\color{brown}
\begin{devext}[#1]\em
\noindent
\begin{itemize}
\item[]\hspace{-3ex}\textbf{Name:} #2
\item[]\hspace{-3ex}\textbf{Extends\ } #3
\item[]\hspace{-3ex}\textbf{New definitions and theorems:}
\end{itemize}
#4
\end{devext}
}

\newenvironment{theory-trans-def}[6]
{
\color{brown}
\begin{thytransdef}[#1]\em
\noindent
\begin{itemize}
\item[]\hspace{-3ex}\textbf{Name:} #2
\item[]\hspace{-3ex}\textbf{Source theory:} #3
\item[]\hspace{-3ex}\textbf{Target theory:} #4
\item[]\hspace{-3ex}\textbf{Base type mapping:}
\end{itemize}
#5
\begin{itemize}
\item[]\hspace{-3ex}\textbf{Constant mapping:}
\end{itemize}
#6
\end{thytransdef}
}

\newenvironment{theory-trans-ext}[7]
{
\color{brown}
\begin{thytransext}[#1]\em
\noindent
\begin{itemize}
\item[]\hspace{-3ex}\textbf{Name:} #2
\item[]\hspace{-3ex}\textbf{Extends\ } #3
\item[]\hspace{-3ex}\textbf{New source theory:} #4
\item[]\hspace{-3ex}\textbf{New target theory:} #5
\item[]\hspace{-3ex}\textbf{New base type mapping:}
\end{itemize}
#6
\begin{itemize}
\item[]\hspace{-3ex}\textbf{New constant mapping:}
\end{itemize}
#7
\end{thytransext}
}

\newenvironment{dev-trans-def}[6]
{
\color{brown}
\begin{devtransdef}[#1]\em
\noindent
\begin{itemize}
\item[]\hspace{-3ex}\textbf{Name:} #2
\item[]\hspace{-3ex}\textbf{Source development:} #3
\item[]\hspace{-3ex}\textbf{Target development:} #4
\item[]\hspace{-3ex}\textbf{Base type mapping:}
\end{itemize}
#5
\begin{itemize}
\item[]\hspace{-3ex}\textbf{Constant mapping:}
\end{itemize}
#6
\end{devtransdef}
}

\newenvironment{dev-trans-ext}[6]
{
\color{brown}
\begin{devtransext}[#1]\em
\noindent
\begin{itemize}
\item[]\hspace{-3ex}\textbf{Name:} #2
\item[]\hspace{-3ex}\textbf{Extends\ } #3
\item[]\hspace{-3ex}\textbf{New source development:} #4
\item[]\hspace{-3ex}\textbf{New target development:} #5
\end{itemize}
\begin{itemize}
\item[]\hspace{-3ex}\textbf{New defined constant mapping:}
\end{itemize}
#6
\end{devtransext}
}

\newenvironment{def-transport}[7]
{
\color{brown}
\begin{deftransport}[#1]\em
\noindent
\begin{itemize}
\item[]\hspace{-3ex}\textbf{Name:} #2
\item[]\hspace{-3ex}\textbf{Development morphism:} #3
\item[]\hspace{-3ex}\textbf{Definition:}
\end{itemize}
#4
\begin{itemize}
\item[]\hspace{-3ex}\textbf{Transported definition:}
\end{itemize}
#5
\begin{itemize}
\item[]\hspace{-3ex}\textbf{New target development:} #6
\item[]\hspace{-3ex}\textbf{New development translation:} #7
\end{itemize}
\end{deftransport}
}

\newenvironment{thm-transport}[6]
{
\color{brown}
\begin{thmtransport}[#1]\em
\noindent
\begin{itemize}
\item[]\hspace{-3ex}\textbf{Name:} #2
\item[]\hspace{-3ex}\textbf{Development morphism:} #3
\item[]\hspace{-3ex}\textbf{Theorem:}
\end{itemize}
#4
\begin{itemize}
\item[]\hspace{-3ex}\textbf{Transported theorem:}
\end{itemize}
#5
\begin{itemize}
\item[]\hspace{-3ex}\textbf{New target development:} #6
\end{itemize}
\end{thmtransport}
}



\newcommand{\axNote}[1]{\hfill (\text{#1})}
\newcommand{\axNoteNL}[1]{\\ \text{} \axNote{#1}}

\newcommand{\cInjB}[1]{\mName{INJ2}(#1)}


\title{Alo\color{lightgray}{alonzo}\color{black}nzo}
\author{Christopher Schankula, Student ID: 400026650, MacID: schankuc\\ Dennis Yankovsky, Student ID: 400189961, MacID: yankovsd\\Group 1 (``We are the Self-Referentialists'')}
\date{December 2022}

\begin{document}

\maketitle

\section{Introduction}
The goal of this project is to formalize Alonzo types and expressions, the notion of free and bound variables,
and the notion of substitution, in Alonzo.

\section{New Notational Definitions}
The following new notational definitions will be used throughout this document:

\noindent$\mName{INJ2}_{\cFunTyX{\cFunTyB{\alpha}{\beta}{\gamma}}{\cB}} \text{ stands for }
\cFunAbs{f}{\cFunTyBX{\alpha}{\beta}{\gamma}}{\cInj{\cDefDesX{g}{\cFunTyX{\alpha \times \beta}{\gamma}}{\cForallBX{a}{\alpha}{b}{\beta}{\cFunAppX{g}{(a,b)} \simeq \cFunAppBX{f}{a}{b}}}}}
$\\

\noindent$\mName{-_{\cFunTyBX{\{\alpha\}}{\{\alpha\}}{\{\alpha\}}}} \text{ stands for } \cFunAbsX{A}{\cSetTy{\alpha}}{\cFunAbsX{B}{\cSetTy{\alpha}}{A \cap \overline{B}}}$

\newpage
\section{Modules}
\begin{theory-def}
% Human-Readable Name
{Strings}
% Name
{$\mName{STR}$.}
% Base Types
{$S$, $C$.}
% Constants
% S = string
% C = char
{$\mName{Stringify}_{\cFunTyX{\cBaseTy{C}}{\cBaseTy{S}}}$, $\mName{Append}_{\cFunTyBX{\cBaseTy{C}}{\cBaseTy{S}}{\cBaseTy{S}}}$,
$A_C, B_C, C_C, ... , Z_C$.}
% Axioms
{}
{\be
\item $\cDistinctL{A,B,C,...,Z} \axNote{all characters are distinct}$
\item $\forall c_1, c_2: \mName{C}, s: S\ .\ \mName{Stringify } c_1 \neq \mName{Append } c_2\ s \axNoteNL{no confusion between Stringify and Append constructors}$
\item $\cInj{\mName{Stringify}} \axNote{no confusion}$
\item $\cTotal{\mName{Stringify}} \axNote{no confusion}$
\item $\cInjB{\mName{Append}} \axNote{no confusion}$
\item $\cTotalB{\mName{Append}} \axNote{no confusion}$
\item $\cForallX{p}{\cFunTyX{S}{\cB}}{\cForall{c}{C}{\cFunAppX{p} {(\mName{Stringify}\ c)}}} \land\\
%%%%%%%%%%%%%%%%%%%
\cForallB{c}{C}{s}{S}{\cFunAppX{p}{s} \Implies
\cFunAppX{p}{\cFunAppB{\mName{Append}}{c}{s}}}\\
%%%%%%%%%%%%%%%%%%%
\Implies \cForallX{s}{S}{\cFunAppX {p}{s}}
\axNote{induction principle for strings}$
\ee}
\end{theory-def}

% \begin{theory-def}
%     {Strings}
%     {$\mName{STR}$.}
%     {$N$, $C$}
%     {$\mName{Nums}_{\{N\}},$
%     $A_C, B_C, C_C, ... , Z_C,$
%     ${\sf strings}_{\{N \rightarrow C\}}$}
%     {}
%     {
%     \be
%         \item ${\sf strings} = $
%     \ee
%     }
% \end{theory-def}

\begin{theory-ext}
% Human-Readable Name
{Types}
% Name
{$\mName{TY}$.}
% Extends
{$\mName{STR}$.}
% New base types
{$AloTy$.}
% New constants
{$\\\mName{BoolAloTy}_{AloTy},\\
\mName{BaseAloTy}_{\cFunTyX{S}{AloTy}},\\
\mName{FunAloTy}_{\cFunTyBX{AloTy}{AloTy}{AloTy}},\\
\mName{ProdAloTy}_{\cFunTyBX{AloTy}{AloTy}{AloTy}},\\
%%%%% Functions for assigning types %%%%%
\mName{isBoolAloTy}_{\cFunTyX{AloTy}{\cB}},\\
\mName{getBaseAloTy}_{\cFunTyX{AloTy}{S}},\\
\mName{getFunAloTy}_{\cFunTyX{AloTy}{AloTy \times AloTy}},\\
\mName{getProdAloTy}_{\cFunTyX{AloTy}{AloTy \times AloTy}}
$}
% Axiom N/A
{}
% Axioms
{
\be
\item $\forall a, b, c, d: AloTy, s: S\ .\ \cDistinctL{\mName{BoolAloTy}, \mName{BaseAloTy s}, \\\mName{FunAloTy a b}, \mName{ProdAloTy c d}} \axNoteNL{no confusion}$

\item $\cInj{\mName{BaseAloTy}} \axNote{no confusion}$

\item $\cInjB{\mName{FunAloTy}} \axNote{no confusion}$

\item $\cInjB{\mName{ProdAloTy}} \axNote{no confusion}$

\item $\cTotal{\mName{BaseAloTy}} \axNote{all strings have a corresponding base type}$

\item $\cTotalB{\mName{FunAloTy}} \axNote{functions accept all types}$

\item $\cTotalB{\mName{ProdAloTy}} \axNote{products accept all types}$

\item $\mName{isBoolAloTy} = \cFunAbs{t}{AloTy}{t = \mName{BoolAloTy}} \axNote{definition of isBoolAloTy}$

\item $\cForallX{p}{\cFunTyX{AloTy}{\cB}}{\cFunApp{p}
{\mName{BoolAloTy}} \land \\
%%%%%%%%%%%%%%%%%%%
\cForall{s}{S}{\cFunAppX{p} {(\mName{BaseAloTy}\ s)}} \land\\
%%%%%%%%%%%%%%%%%%%
\cForallB{t}{AloTy}{u}{AloTy}{\cFunAppX{p}{t} \land
\cFunAppX{p}{u} \Implies
\cFunAppX{p}{\cFunAppB{\mName{FunAloTy}}{t}{u}}} \land\\
%%%%%%%%%%%%%%%%%%%
\cForallB{t}{AloTy}{u}{AloTy}{\cFunAppX{p}{t} \land
\cFunAppX{p}{u} \Implies
\cFunAppX{p}{\cFunAppB{\mName{ProdAloTy}}{t}{u}}} 
%%%%%%%%%%%%%%%%%%%
\\\Implies \cForallX{t}{AloTy}{\cFunAppX {p}{t}}}
\axNoteNL{induction principle for AloTy}$

% \item $\cForallX{t}{AloTy}{t \neq \mName{BoolAloTy} \Implies \neg \ \mName{isBoolAloTy}\ t} \axNoteNL{only BoolAloTys are booleans}$

% \item $\cForallX{s}{AloTy}{\mName{getBaseAloTy (BaseAloTy s)} = s} \axNoteNL{base types are extractable}$

% \item $\cForallBX{s}{AloTy}{t}{AloTy}{s \neq \mName{BaseAloTy t} \Implies \cIsUndefX{(\mName{getBaseAloTy}\ s)}} \axNoteNL{base type extractability law}$

\item $\mName{getFunAloTy} = \cFunAbs{t}{AloTy} {\cDefDesX{s}{\mName{S}}{t = \mName{BaseAloTy}\ s
    }}
    \axNoteNL{base type extractability law}
    $

% \item $\cForallBX{s}{AloTy}{t}{AloTy}{\mName{getFunAloTy (FunAloTy a b)} = (a, b)} \axNoteNL{function types are extractable}$

% \item $\cForallCX{s}{AloTy}{t}{AloTy}{u}{AloTy}{s \neq \mName{FunAloTy t u} \Implies \cIsUndefX{(\mName{getFunAloTy}\ s)}} \axNoteNL{function type extractbility law}$
% spa
% \item $\mName{getFunAloTy} = 
% \cFunAbsX{t}{AloTy}{
% \cDefDesX{p}{AloTy\times AloTy}{\cForsomeB{u}{AloTy}
%         {v}{AloTy}
%         {t = \mName{FunAloTy} (u, v) \land p = (u, v)}
%     }}
%     \axNoteNL{function type extractability law}
%     $
    
\item $\mName{getFunAloTy} = 
\cFunAbs{t}{AloTy}{
\cDefDesX{p}{AloTy\times AloTy}{
        {t = \mName{FunAloTy}\ p}
    }}
    \axNoteNL{function type extractability law}
    $

% \item $\cForallBX{a}{AloTy}{b}{AloTy}{\mName{getProdAloTy (ProdAloTy a b)} = (a, b)} \axNoteNL{product types are extractable}$

% \item $\cForallX{s}{AloTy}{(\cIfX{\cForsomeB{t}{AloTy}
%         {u}{AloTy}
%         {s = \mName{ProdAloTy t u}}
%     }
%     {\cIsDefX{(\mName{getProdAloTy}\ s)}}
%     {\cIsUndefX{(\mName{getProdAloTy}\ s)}})}
%     \axNoteNL{product type extractability law}
%     $

\item $\mName{getProdAloTy} = 
\cFunAbs{t}{AloTy}{
\cDefDesX{p}{AloTy\times AloTy}{
        {t = \mName{ProdAloTy}\ p}
    }}
    \axNoteNL{product type extractability law}
    $

% \item $\mName{getProdAloTy} = \cDefDesX{p}{AloTy\times AloTy}{\cForsomeB{t}{AloTy}
%         {u}{AloTy}
%         {s = \mName{ProdAloTy} (t, u) \land p = (t, u)}
%     }
%     \axNoteNL{product type extractability law}
%     $
    
\ee
}
\end{theory-ext}

% Gödel's playground
% ProdTy BaseTy(ab) BaseTy(xy)

% g(t) = Godel number of t

% BoolTy (2), BaseTy (3), FunTy (5), ProdTy (7)

% g(ProdTy ab xy) = 7 ^ (2^g(ab) * 3^g(xy))

\begin{theory-ext}
% Human-Readable Name
{Pre - Expressions}
% Name
{$\mName{PREXPR}$.}
% Extends
{$\mName{TY}$.}
% New base types
{$Expr$.}
% New constants
{$\\\mName{VarExpr}_{\cFunTyBX{S}{AloTy}{Expr}},\\ \mName{ConstExpr}_{\cFunTyBX{S}{AloTy}{Expr}},\\ \mName{EqExpr}_{\cFunTyBX{Expr}{Expr}{Expr}},\\ \mName{FunAppExpr}_{\cFunTyBX{Expr}{Expr}{Expr}},\\ \mName{FunAbsExpr}_{\cFunTyBX{Expr}{Expr}{Expr}},\\ \mName{DefDesExpr}_{\cFunTyBX{Expr}{Expr}{Expr}},\\ \mName{OrdPairExpr}_{\cFunTyBX{Expr}{Expr}{Expr}}
$}
% Axiom N/A
{}
% Axioms
{
\be
\item $\forall e_1, e_2, e_3, e_4, e_5, e_6, e_7, e_8, e_9,
e_{10}: Expr, t_1, t_2: AloTy, s_1, s_2: S\ .\ \\
\cDistinctL{\cFunAppBX{\mName{VarExpr}}{s_1}{t_1},
\cFunAppBX{\mName{ConstExpr}}{s_2}{t_2},
\cFunAppBX{\mName{EqExpr}}{e_1}{e_2},\\
\cFunAppBX{\mName{FunAppExpr}}{e_3}{e_4},
\cFunAppBX{\mName{FunAbsExpr}}{e_5}{e_6},
\cFunAppBX{\mName{DefDesExpr}}{e_7}{e_8},\\
\cFunAppBX{\mName{OrdPairExpr}}{e_9}{e_{10}}} \axNoteNL{no confusion}$

\item $\cInjB{\mName{VarExpr}} \axNote{no confusion}$
\item $\cInjB{\mName{ConstExpr}} \axNote{no confusion}$
\item $\cInjB{\mName{EqExpr}} \axNote{no confusion}$
\item $\cInjB{\mName{FunAppExpr}} \axNote{no confusion}$
\item $\cInjB{\mName{FunAbsExpr}} \axNote{no confusion}$
\item $\cInjB{\mName{DefDesExpr}} \axNote{no confusion}$
\item $\cInjB{\mName{OrdPairExpr}} \axNote{no confusion}$
\item $\cTotalB{\mName{VarExpr}} \axNote{no confusion}$
\item $\cTotalB{\mName{ConstExpr}} \axNote{no confusion}$
\item $\cTotalB{\mName{EqExpr}} \axNote{no confusion}$
\item $\cTotalB{\mName{FunAppExpr}} \axNote{no confusion}$
\item $\cTotalB{\mName{FunAbsExpr}} \axNote{no confusion}$
\item $\cTotalB{\mName{DefDesExpr}} \axNote{no confusion}$
\item $\cTotalB{\mName{OrdPairExpr}} \axNote{no confusion}$
\item $\cForallX{p}{\cFunTyX{Expr}{\cB}}{ \cForallB{s}{S}{t}{AloTy}{\cFunAppX{p}
{(\cFunAppBX{\mName{VarExpr}}{s}{t})}} \land \\ \cForallB{s}{S}{t}{AloTy}{\cFunAppX{p}
{(\cFunAppBX{\mName{ConstExpr}}{s}{t})}} \land \\
%%%%%%%%%%%%%%%%
\cForall{e_1,e_2}{Expr}{\cFunAppX{p}{e_1} \land \cFunAppX{p}{e_2} \Implies
\cFunAppX{p}{\cFunApp{\mName{EqExpr}}{e_1 e_2}}} \land \\
%%%%%%%%%%%%%%%%
\cForall{e_1,e_2}{Expr}{\cFunAppX{p}{e_1} \land \cFunAppX{p}{e_2} \Implies
\cFunAppX{p}{\cFunApp{\mName{FunAppExpr}}{e_1 e_2}}} \land \\
%%%%%%%%%%%%%%%%
\cForall{e_1,e_2}{Expr}{\cFunAppX{p}{e_1} \land \cFunAppX{p}{e_2} \Implies
\cFunAppX{p}{\cFunApp{\mName{FunAbsExpr}}{e_1 e_2}}} \land \\
%%%%%%%%%%%%%%%%
\cForall{e_1,e_2}{Expr}{\cFunAppX{p}{e_1} \land \cFunAppX{p}{e_2} \Implies
\cFunAppX{p}{\cFunApp{\mName{DefDesExpr}}{e_1 e_2}}} \land \\
%%%%%%%%%%%%%%%%
\cForall{e_1,e_2}{Expr}{\cFunAppX{p}{e_1} \land \cFunAppX{p}{e_2} \Implies
\cFunAppX{p}{\cFunApp{\mName{OrdPairExpr}}{e_1 e_2}}} 
%%%%%%%%%%%%%%%%
\\\Implies \cForallX{e}{Expr}{\cFunAppX {p}{e}}}
\axNoteNL{induction principle for Expr}$
\ee
}
\end{theory-ext}

\begin{theory-ext}
% Human-Readable Name
{Expressions}
% Name
{$Expr$.}
% Extends
{$\mName{PREXPR}$.}
% New base types
{$Expr$.}
% New constants
{$
\\\mName{hasAloTy}_{\cFunTyX{Expr}{AloTy}}, \\
\mName{VE}_{\{Expr\}}, \\
\mName{sane}_{\cFunTy{Expr}{Expr}},\\ 
\mName{VarVExpr}_{\cFunTyBX{S}{AloTy}{Expr}},\\
\mName{ConstVExpr}_{\cFunTyBX{S}{AloTy}{Expr}},\\
\mName{EqVExpr}_{\cFunTyBX{Expr}{Expr}{Expr}},\\ 
\mName{FunAppVExpr}_{\cFunTyBX{Expr}{Expr}{Expr}},\\ \mName{FunAbsVExpr}_{\cFunTyBX{Expr}{Expr}{Expr}},\\ \mName{DefDesVExpr}_{\cFunTyBX{Expr}{Expr}{Expr}},\\ \mName{OrdPairVExpr}_{\cFunTyBX{Expr}{Expr}{Expr}}
$}
% Axiom N/A
{}
% Axioms
{
\be
\item $\mName{hasAloTy} = \cFunAbsX{e}{Expr}
    {
    \cDefDesX{t}{AloTy}{}\\
        \cForsomeUnique{s}{S}{e = \cFunAppB{\mName{VarExpr}}{s}{t}}\\
        %%%%%%%%%%%%%%%%
        \lor \cForsomeUnique{s}{S}{e = \cFunAppB{\mName{ConstExpr}}{s}{t}}\\
        %%%%%%%%%%%%%%%%
        \lor \cForsomeUnique{e_1,e_2}{Expr}{e = (\cFunAppBX{\mName{EqExpr}}{e_1}{e_2}) \land
        \cFunAppX{\mName{hasAloTy}}{e_1} = \cFunAppX{\mName{hasAloTy}}{e_2} \land \cFunAppX{\mName{isBoolAloTy}}{t}} \\
        %%%%%%%%%%%%%%%%
        \lor \cForsomeUniqueB{e_1, e_2}{Expr}{t'}{AloTy}{e = 
        (\cFunAppBX{\mName{FunAppExpr}}{e_1}{e_2})
        \land \cFunAppX{\mName{hasAloTy}}{e_1} =\\\cFunAppB{\mName{FunAloTy}}{t'}{t} \land
        \cFunAppX{\mName{hasAloTy}}{e_2} = t'} \\
        %%%%%%%%%%%%%%%%
        \lor \cForsomeUniqueC{e'}{Expr}{t'}{AloTy}{s}{S}{e = 
        \cFunAppB{\mName{FunAbsExpr}}{\cFunAppB{\mName{VarExpr}}{s}{t'}}{e'}
        \land t = \cFunAppB{\mName{FunAloTy}}{t'}{\cFunApp{\mName{hasAloTy}}{e'}}} \\
        %%%%%%%%%%%%%%%%
        \lor \cForsomeUniqueB{e'}{Expr}{s}{S}{e = \cFunAppB{\mName{DefDesExpr}}{
        \cFunAppB{\mName{VarExpr}}{s}{t}}{e'} \land
        \neg \mName{isBoolAloTy}(t) \land
        \cFunAppX{\mName{isBoolAloTy}}{} {\cFunApp{\mName{hasAloTy}}{e'}}}\\
        %%%%%%%%%%%%%%%%
        \lor \cForsomeUnique{e_1,e_2}{Expr}{(e = \cFunAppBX{\mName{OrdPairExpr}}{e_1}{e_2}) \land t = \cFunAppBX{\mName{ProdAloTy}}{\cFunApp{\mName{hasAloTy}}{e_1}}{}\\{\cFunApp{\mName{hasAloTy}}{e_2}}} \axNoteNL{definition of hasAloTy}\\
        %%%%%%%%%%%%%%%%
    }$
\item $\mName{VE}_{\cSetTy{Expr}} = {\sf dom}(\mName{hasAloTy}) \axNote{definition of valid expressions}$
\item $\mName{sane} = \cFunAbsQTyX{e}{\mName{VE}_{\cSetTy{Expr}}}{e} \axNote{definition of Expr sanitization function}$
\item $\mName{VarVExpr} = \mName{VarExpr} \axNote{smart constructor for variables}$
\item $\mName{ConstVExpr} = \mName{ConstExpr} \axNote{smart constructor for constants}$
\item $\mName{EqVExpr} = \cFunAbsQTyX{e_1, e_2}{\mName{VE}_{\cSetTy{Expr}}}{\cFunAppX{\mName{sane}}{\cFunAppB{\mName{EqExpr}}{e_1}{e_2}}} \axNoteNL{smart constructor for equality}$
\item $\mName{FunAppVExpr} = \cFunAbsQTyX{e_1, e_2}{\mName{VE}_{\cSetTy{Expr}}}{\cFunAppX{\mName{sane}}{\cFunAppB{\mName{FunAppExpr}}{e_1}{e_2}}} \axNoteNL{smart constructor for function application}$
\item $\mName{FunAbsVExpr} = \cFunAbsQTyX{e_1, e_2}{\mName{VE}_{\cSetTy{Expr}}}{\cFunAppX{\mName{sane}}{\cFunAppB{\mName{FunAbsExpr}}{e_1}{e_2}}} \axNoteNL{smart constructor for function abstraction}$
\item $\mName{DefDesVExpr} = \cFunAbsQTyX{e_1, e_2}{\mName{VE}_{\cSetTy{Expr}}}{\cFunAppX{\mName{sane}}{\cFunAppB{\mName{DefDesExpr}}{e_1}{e_2}}} \axNoteNL{smart constructor for definite description}$
\item $\mName{OrdPairVExpr} = \cFunAbsQTyX{e_1, e_2}{\mName{VE}_{\cSetTy{Expr}}}{\cFunAppX{\mName{sane}}{\cFunAppB{\mName{OrdPairExpr}}{e_1}{e_2}}} \axNoteNL{smart constructor for ordered pair}$
\ee
}
\end{theory-ext}

\begin{theory-ext}
% Human-Readable Name
{Free and Bound Variables}
% Name
{$\mName{FBVAR}$.}
% Extends
{$\mName{Expr}$.}
% New base types
{N/A}
% New constants
{$\\ \mName{varIsFree}_{\cFunTyBX{Expr}{Expr}{\cB}},\\
\mName{freeVars}_{\cFunTyX{Expr}{\{Expr\}}},\\
\mName{varsInExpr}_{\cFunTyX{Expr}{\{Expr\}}},\\
\mName{varIsBound}_{\cFunTyBX{Expr}{Expr}{\cB}},\\
\mName{boundVars}_{\cFunTyX{Expr}{\{Expr\}}}
$}
% Axiom N/A
{}
% Axioms
{
\be
    % free function
    \item $\mName{freeVars} = \cFunAbsX{e}{Expr}
    {
    \cDefDesX{x}{\mName{\{Expr\}}}{}\\ % Variables
        \cForsomeUniqueB{s}{S}{t}{AloTy}{(e = \cFunAppBX{\mName{VarVExpr}}{s}{t}) \land x =
        \{e\}}\\
        %%%%%%%%%%%%%%%% Constants
        \lor \cForsomeUniqueB{s}{S}{t}{AloTy}{(e = \cFunAppBX{\mName{ConstVExpr}}{s}{t}) \land x =
        \emptyset_{\{Expr\}}}\\
        %%%%%%%%%%%%%%%% Equality
        \lor \cForsomeUnique{e_1,e_2}{Expr}{e = (\cFunAppBX{\mName{EqVExpr}}{e_1}{e_2}) \land
        x = \cFunApp{\mName{freeVars}}{e_1} \cup \cFunApp{\mName{freeVars}}{e_2}} \\
        %%%%%%%%%%%%%%%% Function Application
        \lor \cForsomeUniqueB{e_1, e_2}{Expr}{t'}{AloTy}{e = 
        (\cFunAppBX{\mName{FunAppVExpr}}{e_1}{e_2})
        \land \\
        \mName{} \hspace{1cm} x = \cFunApp{\mName{freeVars}}{e_1} \cup \cFunApp{\mName{freeVars}}{e_2}}\\
        %%%%%%%%%%%%%%%% Function Abstraction
        \lor \cForsomeUnique{v, e_1}{Expr}{e = 
        \cFunAppB{\mName{FunAbsVExpr}}{v}{e_1}
        \land x = \mName{freeVars}(e_1) - \{v\}
        } \\
        %%%%%%%%%%%%%%%% Definite Description
        \lor \cForsomeUnique{v, e_1}{Expr}{e = 
        \cFunAppB{\mName{DefDesVExpr}}{v}{e_1}
        \land x = \mName{freeVars}(e_1) - \{v\}
        }\\
        %%%%%%%%%%%%%%%% Ordered Pair
        \lor \cForsomeUnique{e_1,e_2}{Expr}{(e = \cFunAppBX{\mName{OrdPairVExpr}}{e_1}{e_2}) \land x = \mName{freeVars}(e_1) \cup \mName{freeVars}(e_2)} \axNoteNL{definition of free}\\
        %%%%%%%%%%%%%%%%
    }$
    
    % varsInExpr function
    \item $\mName{varsInExpr} = \cFunAbsX{e}{Expr}
    {
    \cDefDesX{x}{\mName{\{Expr\}}}{}\\ % Variables
        \cForsomeUniqueB{s}{S}{t}{AloTy}{(e = \cFunAppBX{\mName{VarVExpr}}{s}{t}) \land x =
        \{e\}}\\
        %%%%%%%%%%%%%%%% Constants
        \lor \cForsomeUniqueB{s}{S}{t}{AloTy}{(e = \cFunAppBX{\mName{ConstVExpr}}{s}{t}) \land x =
        \emptyset_{\{Expr\}}}\\
        %%%%%%%%%%%%%%%% Equality
        \lor \cForsomeUnique{e_1,e_2}{Expr}{e = (\cFunAppBX{\mName{EqVExpr}}{e_1}{e_2}) \land
        x = \cFunApp{\mName{varsInExpr}}{e_1} \cup \cFunApp{\mName{varsInExpr}}{e_2}} \\
        %%%%%%%%%%%%%%%% Function Application
        \lor \cForsomeUniqueB{e_1, e_2}{Expr}{t'}{AloTy}{e = 
        (\cFunAppBX{\mName{FunAppVExpr}}{e_1}{e_2})
        \land \\
        \mName{} \hspace{1cm} x = \cFunApp{\mName{varsInExpr}}{e_1} \cup \cFunApp{\mName{varsInExpr}}{e_2}}\\
        %%%%%%%%%%%%%%%% Function Abstraction
        \lor \cForsomeUnique{v, e_1}{Expr}{e = 
        \cFunAppB{\mName{FunAbsVExpr}}{v}{e_1}
        \land x = \mName{varsInExpr}(e_1)
        } \\
        %%%%%%%%%%%%%%%% Definite Description
        \lor \cForsomeUnique{v, e_1}{Expr}{e = 
        \cFunAppB{\mName{DefDesVExpr}}{v}{e_1}
        \land x = \mName{varsInExpr}(e_1)
        }\\
        %%%%%%%%%%%%%%%% Ordered Pair
        \lor \cForsomeUnique{e_1,e_2}{Expr}{(e = \cFunAppBX{\mName{OrdPairVExpr}}{e_1}{e_2}) \land x = \mName{varsInExpr}(e_1) \cup \mName{varsInExpr}(e_2)} \axNoteNL{definition of varsInExpr}\\
        %%%%%%%%%%%%%%%%
    }$
    
    \item $\mName{boundVars} = \cFunAbs{e}{Expr}{(\mName{varsInExpr} \: e) - (\mName{freeVars} \: e)}\axNote{set of free}$
    
    \item $\mName{varIsBound} = 
    \cFunAbs{v}{Expr}{\cFunAbs{e}{Expr}{
    \cForsomeUniqueB{s}{S}{t}{AloTy}{(v = \mName{VarVExpr} \: s \: t) \mapsto (v \in
    (\mName{boundVars} \: e)) \: | \: \bot}}} \axNote{bound variable predicate}$
    
    \item $\mName{varIsFree} =
    \cFunAbs{v}{Expr}{\cFunAbs{e}{Expr}{
    \cForsomeUniqueB{s}{S}{t}{AloTy}{(v = \mName{VarVExpr} \: s \: t) \mapsto (v \in
    (\mName{freeVars} \: e)) \: | \: \bot}}} \axNote{free variable predicate}$
\ee
}
\end{theory-ext}

\begin{theory-ext}
% Human-Readable Name
{Substitution}
% Name
{$\mName{SUBS}$.}
% Extends
{$\mName{FBVAR}$.}
% New base types
{N/A}
% New constants
{$\\ \mName{sub}_{\cFunTyCX{Expr}{Expr}{Expr}{Expr}}
$}
% Axiom N/A
{}
% Axioms
{
\be
    \item $\mName{sub} = \cFunAbsX{A}{Expr}
    {\cFunAbsX{x}{Expr}
    {\cFunAbsX{B}{Expr}
    {\cDefDesX{C}{Expr}
    {\\ (\cFunAppX{\mName{hasAloTy}}{A} = \cFunAppX{\mName{hasAloTy}}{x}) \land \cForsomeB{s}{S}{t}{AloTy}{x = \cFunAppBX{\mName{VarVExpr}}{s}{t}}
    %%%%%%%%%%%%%%
    % Expression B is just variable x
    \\ \land (
    \cForsomeUniqueB{s}{S}{t}{AloTy}{B = \cFunAppB{\mName{VarVExpr}}{s}{t} \land B = x \land C = A}
    %%%%%%%%%%%%%%
    % Expression B is a variable, but not variable x
    \\ \lor
    \cForsomeUniqueB{s}{S}{t}{AloTy}{B = \cFunAppB{\mName{VarVExpr}}{s}{t} \land B \neq x \land C = B}
    %%%%%%%%%%%%%%
    % Expression B is a constant
    \\ \lor
    \cForsomeUniqueB{s}{S}{t}{AloTy}{B = \cFunAppB{\mName{ConstVExpr}}{s}{t} \land C = B}
    %%%%%%%%%%%%%%
    % Expression B is an equality
    \\ \lor
    \cForsomeUnique{e_1,e_2}{Expr}{B = \cFunAppB{\mName{EqVExpr}}{e_1}{e_2} \land
    \\ \mName{} \hspace{1cm} C = \cFunAppBX{\mName{EqVExpr}}{\cFunAppC{\mName{sub}}{A}{x}{e_1}}{\cFunAppC{\mName{sub}}{A}{x}{e_2}}}
    %%%%%%%%%%%%%%
    % Expression B is a function application
    \\ \lor
    \cForsomeUnique{e_1,e_2}{Expr}{B = \cFunAppB{\mName{FunAppVExpr}}{e_1}{e_2} \land
    \\ \mName{} \hspace{1cm} C = \cFunAppBX{\mName{FunAppVExpr}}{\cFunAppC{\mName{sub}}{A}{x}{e_1}}{\cFunAppC{\mName{sub}}{A}{x}{e_2}}}
    %%%%%%%%%%%%%%
    % Expression B is a function abstraction
    \\ \lor
    \cForsomeUnique{e_1,e_2}{Expr}{B = \cFunAppB{\mName{FunAbsVExpr}}{e_1}{e_2} \land e_1 = x \land C = B}
    %%%%%%%%%%%%%%
    % Expression B is a function abstraction (variables different)
    \\ \lor
    \cForsomeUnique{e_1,e_2}{Expr}{B = \cFunAppB{\mName{FunAbsVExpr}}{e_1}{e_2} \land e_1 \not= x \land
    \\ \mName{} \hspace{1cm} C = \cFunAppBX{\mName{FunAbsVExpr}}{e_1}{\cFunAppC{\mName{sub}}{A}{x}{e_2}}}
    %%%%%%%%%%%%%%
    % Expression B is a definite description
    \\ \lor
    \cForsomeUnique{e_1,e_2}{Expr}{B = \cFunAppB{\mName{DefDesVExpr}}{e_1}{e_2} \land e_1 = x \land C = B}
    %%%%%%%%%%%%%%
    % Expression B is a definite description (variables different)
    \\ \lor
    \cForsomeUnique{e_1,e_2}{Expr}{B = \cFunAppB{\mName{DefDesVExpr}}{e_1}{e_2} \land e_1 \not= x \land
    \\ \mName{} \hspace{1cm} C = \cFunAppBX{\mName{DefDesVExpr}}{e_1}{\cFunAppC{\mName{sub}}{A}{x}{e_2}}}
    %%%%%%%%%%%%%%
    % Expression B is an ordered pair
    \\ \lor
    \cForsomeUnique{e_1,e_2}{Expr}{B = \cFunAppB{\mName{OrdPairVExpr}}{e_1}{e_2} \land
    \\ \mName{} \hspace{1cm} C = \cFunAppBX{\mName{OrdPairVExpr}}{\cFunAppC{\mName{sub}}{A}{x}{e_1}}{\cFunAppC{\mName{sub}}{A}{x}{e_2}}}
    )
    }}}} \axNoteNL{definition of the substitution function}$ 
    
    
    % $\mName{sub} = \cFunAbsCX{A}{Expr}
    % {
    % \cDefDesX{x}{\mName{\{Expr\}}}{}\\ % Variables
    %     \cForsomeUniqueB{s}{S}{t}{AloTy}{(e = \cFunAppBX{\mName{VarVExpr}}{s}{t}) \land x =
    %     \{e\}}\\
    %     %%%%%%%%%%%%%%%% Constants
    %     \lor \cForsomeUniqueB{s}{S}{t}{AloTy}{(e = \cFunAppBX{\mName{ConstVExpr}}{s}{t}) \land x =
    %     \emptyset_{\{Expr\}}}\\
    %     %%%%%%%%%%%%%%%% Equality
    %     \lor \cForsomeUnique{e_1,e_2}{Expr}{e = (\cFunAppBX{\mName{EqVExpr}}{e_1}{e_2}) \land
    %     x = \cFunApp{\mName{varsInExpr}}{e_1} \cup \cFunApp{\mName{varsInExpr}}{e_2}} \\
    %     %%%%%%%%%%%%%%%% Function Application
    %     \lor \cForsomeUniqueB{e_1, e_2}{Expr}{t'}{AloTy}{e = 
    %     (\cFunAppBX{\mName{FunAppVExpr}}{e_1}{e_2})
    %     \land \\
    %     \mName{} \hspace{1cm} x = \cFunApp{\mName{varsInExpr}}{e_1} \cup \cFunApp{\mName{varsInExpr}}{e_2}}\\
    %     %%%%%%%%%%%%%%%% Function Abstraction
    %     \lor \cForsomeUnique{v, e_1}{Expr}{e = 
    %     \cFunAppB{\mName{FunAbsVExpr}}{v}{e_1}
    %     \land x = \mName{varsInExpr}(e_1)
    %     } \\
    %     %%%%%%%%%%%%%%%% Definite Description
    %     \lor \cForsomeUnique{v, e_1}{Expr}{e = 
    %     \cFunAppB{\mName{DefDesVExpr}}{v}{e_1}
    %     \land x = \mName{varsInExpr}(e_1)
    %     }\\
    %     %%%%%%%%%%%%%%%% Ordered Pair
    %     \lor \cForsomeUnique{e_1,e_2}{Expr}{(e = \cFunAppBX{\mName{OrdPairVExpr}}{e_1}{e_2}) \land x = \mName{varsInExpr}(e_1) \cup \mName{varsInExpr}(e_2)} \axNoteNL{definition of varsInExpr}\\
    %     %%%%%%%%%%%%%%%%
    % }$
\ee
}
\end{theory-ext}

\section{Conclusions}
Implementing Alonzo's syntax in Alonzo was a wonderful exercise giving the team a much greater understanding of
not only Alonzo but also how to use it to define inductively-defined sets as types and their properties. 

\subsection{Importance of Establishing Conventions}
Initially,
it was found that one of the most difficult parts of starting the project was being able to converse about the
Alonzo system itself, as the team members easily got confused about whether one was discussing properties (e.g.
types, expressions) of Alonzo itself or the embedding of Alonzo in Alonzo. Thus, it was decided that we would 
refer to Alonzo itself as ``outer Alonzo'' and our embedding of Alonzo inside Alonzo as ``inner Alonzo'' or 
``Alo'' for short.

\subsection{Use of Quasitypes}
The use of quasitypes to restrict to expressions to well-defined ones was a very interesting use of this concept
in Alonzo. It allowed us to reuse the logic from the $\mName{hasAloTy}$ function for defining the quasitype very
easily, which ultimately made the smart constructors very easy to implement, if repetitive. The use of sorts in
$\mName{AlonzoS}$ should be explored to make this much more type-safe.

\subsection{Recommendations for Alonzo}
Given the number of inductively-designed sets we needed for the project, there were a lot of extra axioms just
to ensure ``no confusion'' and ``no junk''. While this was very instructive as to how inductively-designed sets
like Haskell's algebaric data types work, it was also very repetitive. Thus, the first recommendation would be to
create an extension of Alonzo allowing one to create these algebraic data types automatically, providing the
axioms for distinctness, totality, injectivity and the induction principle.

Secondly, the use of pattern-matching in the textbook makes writing inductive definitions easier, but this is
not natively supported in Alonzo. Exploring the addition of pattern-matching on inductively-defined sets is
another recommendation of the team. Currently, it can be ``faked'' by using a definite description with a 
disjunction for all the cases. For base cases, this is usually a simple conjunction both matching the input and 
then assigning the output. For recursive cases, we use (optionally, unique) existential quantifiers for 
induction cases to ``extract'' the right variables, then set the variable
bound by the definition description to the correct value for that case, or simply include the output variable
as part of the matched input expression.

Thirdly, the team had to create a few new notational definitions, such as $\mName{INJ2}$ and $-$ for set
subtraction. It's recommended that Alonzo introduces these notational definitions natively, and perhaps others
for injectivity for functions of more than 2 inputs.

The team also created some helper functions for Alonzo $\LaTeX$ notation, such as $\backslash$\texttt{axNote} and 
$\backslash$\texttt{axNoteNL} to name axioms:

\begin{Verbatim}
\newcommand{\axNote}[1]{\hfill (\text{#1})}
\newcommand{\axNoteNL}[1]{\\ \text{} \axNote{#1}}
\end{Verbatim}

These could be incorporated into the official $\LaTeX$ for Alonzo documentation.

% \newpage
% \begin{theory-def}
% % Human-Readable Name
% {Expression}
% % Name
% {$Expr$.}
% % New base types
% {$Expr$, $Z$.}
% % New constants
% {$\text{Num}_{\cFunTyX{Z}{Expr}}, \text{Add}_{\cFunTyBX{Expr}{Expr}{Expr}}, \text{Mult}_{\cFunTyBX{Expr}{Expr}{Expr}}$}
% % Axiom N/A
% {N/A}
% % Axioms
% {}
% \end{theory-def}

% \newpage
% \section {Questions for Bill}
% \begin{itemize}
% \item Can base types be strings or only single characters?
% \item Does the use of functions as constructors make sense? 
% \item How do you represent ``No junk, no confusion'' axiomatically?
% \item How much Alonzo coverage should we aim for? What is the usual scope of projects historically? We were thinking of allowing people to encode types, expressions, languages and theories.
% \item Little theories method or theory extension method?

% \end{itemize}

\end{document}
